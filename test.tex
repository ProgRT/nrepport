\documentclass{nrepport}
\usepackage{fzm}

\title{FZV-LC 2020-08-19 test session}
\author{\nsignf}

%\client{\includegraphics[width=4cm]{/home/nicolas/Documents/FZ/Utils/fzm}}
\client{\fzm[width=3.5cm]}
\model{FZV-LC}
\mode{CMV-VC\newline CMV-PC}
\testtype{Basic observation}
\testdate{2020-08-19}

\def\tp{$T_{pause}$}

\begin{document}
\tpblock
\maketitle
%\testidblock

\tableofcontents

\section*{Key findings}

\begin{itemize}
	\item Inspiratory time ($T_i$) and pause time ($T_{pause}$) settings
		works exactly as expected. \emph{Great job !}
	\item The setting of a \tp\ has a significant impact on the accuracy of
		tidal volume. Pause time increasing tidal volume by as high as
		13 \%. The test protocol will have to take this into account.
	\item Is seems feasible to provide the user with a \emph{plateau
		pressure} readout, which in my opinion would be a major selling
		point.
	\item The lower limit on i:e ratio (1:4) is inadequate. It results
		in imposing a minimal frequency for a set $T_i$, which is neither
		clinically or technically relevant.
	\item Triggering in CMV-VC is still problematic.
	\item Increments of 5 hPa is to gross for PC setting. Increment
		should be 1 hPa.
	\item The ventilator crashed several times in PC. This represent a
		major patient safety issue.
	\item There seems to be a leak at the PEEP valve.
\end{itemize}

\cleardoublepage

\section{Inspiratory time}

With the folowing parameters:

\marginpar{%
	\centering
\begin{tabular}{ll}
	\hline
	PEEP & 11\\
	Tidal volume & 500\\
	Frequency & 8\\
	Inspiratory time & 1.5\\
	Vt calibration factor & 0.95\\
	R & 5\\
	C & 75\\
	\hline
\end{tabular}}

\begin{itemize}
\item There is still a \emph{minimal} I:E of 1:4. I dont see any
clinical justification for a minimal I:E. With a Ti of 1, i am unable
to go below 12 of frequency. Putting a Ti of 1.5 allow me to lower
frequency to 8.

\item In the absence of an expiratory flowmeter, it is impossible to
	visualise the presence or absence of an inspiratory hold on the flow
	signal.
\item Pressure signal sugest there may be a minimal inspiratory
	hold.
\item Visually, the compressor seems to go up as soo as it reache
	the bottom, which is the correct behavior.

\item Vt oscillating between 523 and 554 are read from the
	inspiratory flow sensor.

\item Frequency over 1 min is acurate.

\item The compressor return velocity seems to be dependant on
	inspiratory velocity. It would seems whise that the return (upward)
		be alwais as fast as possible, to be ready for the next
		insuflation.
\end{itemize}

With an added inspiratory hold of 0.7 sec,

\begin{itemize}
	\item Tidal volume increased to 640.
	\item Total inspiratory time is still around 1.5 sec. (validated by
		stopwatch).
	\item setting the Vt calibration factor to .82 allowed to obtain a
		Vt of 506.
\end{itemize}

\section{Triggering}

In CMV-VC, spontaneous respiratory efforts generated negative PEEPS
readouts (as low as -35). It seems likely the reason we have not been
able to trigger more than two consecutive breath cycle.

I suggest that you consider fixing this issue by allowing the user to
input the	\emph{intended PEEP} in the user interface.	This value will
then be use instead of the PEEP readouts for triggering threshold
calculation.

\emph{The user interface will have to make clear the fact
that intended PEEP does not sets the PEEP, which still have to be set
mechanicaly at the PEEP valve}

Another possible solutions would be to exclude cycles with patient
efforts from PEEP readout. But this approach would bring the problem
of differentiating real efforts from changes in PEEP settings.

\section{CMV-PC}

\begin{itemize}
		\begin{figure}
			%\includegraphics[width=\textwidth]{PCcrash}
			\caption{The software crashed several times in CMV-PC mode. The
			ventilation stopped at the time of crash and the ventilator had
			to be reboot.}
		\end{figure}

	\item The software crashed several times while testing CMV-PC mode.
		This prevented us from extensively testing this mode.
	\item The ventilation stopped at the time of softhware crash, making
		it a {\bf major thread for patient safety}.
	\item A PC level of 20 hPa was acurately delivered by the
		ventilator.
	\item The ventilator was unable to deliver a PC of 40 hPa (about 30
		hPa delivered).
	\item This limitation was not caused by a completely compressed
		ambu bag.
	\item Increments of 5 hPa is a to gross setting. Increment should be
		1 hPa.
	\item The compressor had a clumsy movement, during bot compression
		and release (upward), which it does not have in in CMV-VC.
\end{itemize}

\section{User interface and interaction}

\begin{itemize}
	\item The need for two confirmation for every changes of parameter
		is kind of \emph{overkill}-in my opinion- and may leads to user
		confusion.
	\item The flowrate display in the \emph{READOUTS, SETTINGS and
		ALLARMS} tabs is of no use.
	\item Waveforms are missing an identification and unit of
			measurement.
	\item \emph{Stop device} is ambiguous. \emph{Stop ventilation} would
		avoid confusion.
	\item The scales of the graphic display is suboptimal in many ways.
		\emph{Would need a little bit of loving care!}
\end{itemize}

\section{Suggested improvements}

\subsection{Plateau pressure}

\def\pplat{$P_{plat.}$}

The fact that the ventilator is able to deliver and control an
inspiratory hold time bring the possibility to provide plateau
pressure (\pplat) readout. \pplat is the pressure read during an
inspiratory hold (plateau). It is considered a reflect the force
needed to maintain the tidal volume in the lung.

Monitoring and limiting \pplat\ is a major concern in the ventilation
of the most critical patient. Providing this information to the user
would then seems to us like a very relevant feature for your
ventilator.

Technically, it seems as simple as reading the pressure some few
milliseconds before the end of the inspiratory pause time, when such a
pause time have been set by the user.

\subsection{Compressor return velocity}

The compressor return velocity (upward) seems to be set at the same
value as the compression velocity (downward). This means that longer
inspiratory time result in longer compressor return time. And this
means less time where the compressor is up and ready to respond to a
patient trigger.

I then suggest that, no mater the speed used for inspiratory time, the
compressor should always use the highest possible speed for it's
return to his waiting position (within the
limits of what is safe for the hardware).
\end{document}
